\documentclass[uplatex,a4paper,12pt]{jsarticle}

\usepackage[dvipdfmx]{graphicx}
\usepackage{url}
\usepackage{comment} %コメントアウト
\usepackage{otf} %ローマ数字
\usepackage{multirow}

\title{卒業論文レポート}
\author{20RS509 下川 俊彦}
\date{2023年2月9日}

\begin{document}
\maketitle

\section{学生の演習問題解答状況を把握するシステムの開発}\label{motomura}
著者は 16JK137 本村亮、2019年度の卒業論文である。

\subsection{概要}
この論文の概要を書く
論文のまとめの章の丸写しは駄目。自分の文章で書くこと。


\subsection{意見・感想}

この論文に対する自分の意見や感想を述べる。

\section{KIND Wi-Fiの接続情報を用いた出席確認補助システムの改良}\label{inoue}
著者は 16JK012 井上 光洋、2019年度の卒業論文である。

\subsection{概要}
この論文の概要を書く。
論文のまとめの章の丸写しは駄目。自分の文章で書くこと。
    

\subsection{意見・感想}

この論文に対する自分の意見や感想を述べる。


\section{バランス維持訓練ゲームTATSUJINの開発}\label{tatsujin}
著者は 16JK157 吉武 伸太郎、2019年度の卒業論文である。


\subsection{概要}
この論文の概要を書く
論文のまとめの章の丸写しは駄目。自分の文章で書くこと。
    

\subsection{意見・感想}

この論文に対する自分の意見や感想を述べる。

\section{卒業研究について考えたこと}

卒業研究について、この時点で自分が感じたことを自由に書く。
自分が取り組みたいテーマについて書く必要はない。
もちろん、書いても良い。

\ref{motomura}のような研究は難しそうだけど、やってみたい。

とか、なんとか自由に。

\end{document}